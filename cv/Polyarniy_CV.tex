% sudo apt install texlive-latex-base texlive-lang-cyrillic texlive-latex-recommended texlive-latex-extra
% pdflatex Polyarniy_CV.pdf

\documentclass[11pt,oneside]{article}
\usepackage[utf8]{inputenc}
\usepackage[english,russian]{babel}
\usepackage{xcolor}
\usepackage{hyperref}

\usepackage{graphicx}
\usepackage{wrapfig}
\usepackage[absolute,overlay]{textpos}
\graphicspath{ {imgs/} }

\newcommand{\hhref}[2]{\href{#1}{\color{blue}#2}}

\begin{document}

\begin{textblock}{7}(11,1)
    \includegraphics[scale=0.4]{photo.png}
\end{textblock}

\begin{center}
	{\huge Poliarnyi Nikolai}
\end{center}

\vspace{-9pt}
\section*{\textbf{Work Experience}}
\vspace{-9pt}


\begin{description}
  \item[ - Agisoft] \hfill \\
    \textbf{Since April 2016}

    \textbf{Mathematician-Programmer (Team Lead)}

    \hhref{https://en.wikipedia.org/wiki/PhotoScan}{Metashape} developer. R\&D and continous search for points of improvement. Developed a detailed, scale-diverse, fast and scalable (out-of-core and cluster-friendly) surface model reconstruction method (published \hhref{https://www.polarnick.com/static/papers/poliarnyi2021.pdf}{a paper} at the top conference \hhref{http://iccv2021.thecvf.com/}{ICCV 2021}). Student Internship Mentor.

    Computer Vision, Computational Geometry, OpenCL/CUDA/Vulkan, AI/ML
  \item[ - Transas] \hfill \\
    \textbf{October 2014 - March 2016}

    \textbf{Mathematician-Programmer}

    Developed a server that produces 3D landscape reconstruction and true orthophoto stitching from UAVs' data (\hhref{http://polarnick239.github.io/old/cv/Monoceros1.pdf}{presentation}, \hhref{http://polarnick239.github.io/old/cv/Monoceros2.pdf}{second presentation}).

    OpenCV, OpenCL, Python, Cython, Ceres-solver.
  \item[ - Yandex.Money] \hfill \\
    \textbf{February 2014 – October 2014:} Software Developer (Java backend)
  \item[ - DevExperts] \hfill \\
    \textbf{April 2013 – September 2013:} Software Developer (Java backend)

\end{description}


\vspace{-9pt}
\section*{\textbf{Skills}}
\vspace{-9pt}

\begin{itemize}
    \item{\textbf{Computer Vision}}: Structure from Motion, Multiple View Geometry, AI/ML, objects detection/classification/segmentation, magic.
    Better than state of the art depth maps estimation, surface reconstruction, texturing and other algorithms.

    \item{\textbf{Computational geometry, CGAL}}: computations with absolute accuracy, algorithms and structures like Delaunay triangulation.

    \item{\textbf{Vulkan, OpenCL, CUDA, OpenGL, WebGL}}: GPGPU computations, shaders, ray tracing, algorithms profiling/acceleration/adaptation for the GPU. Able to work around bugs in video drivers and compilers...

    \item{\textbf{C++, Python, Java}}
\end{itemize}


\vspace{-9pt}
\section*{\textbf{Activities}}
\vspace{-9pt}

\begin{itemize}
    \item{\textbf{Consultant}}: provides consultation services to companies and startups on topics related to computer vision algorithms and GPU-acceleration.

    \item{\textbf{Photogrammetry course}}: developed Photogrammetry  \hhref{https://compsciclub.ru/courses/photogrammetry/2021-spring/}{course} in Computer Science Club. Teaching it in \hhref{https://math-cs.spbu.ru/}{SPbU}.  \hhref{https://www.youtube.com/@polarnick239/playlists}{Video recordings}. Tasks on \hhref{https://github.com/PhotogrammetryCourse/}{github}.

    \item{\textbf{GPGPU course}}: developed GPGPU OpenCL \hhref{https://compscicenter.ru/courses/video_cards_computation/}{course} in Computer Science Center. \hhref{https://www.youtube.com/watch?v=L79PgDOcVfw&list=PLlb7e2G7aSpTgwAm0GBkvn5XA0NokovJJ}{Video recordings}. Tasks on \hhref{https://github.com/GPGPUCourse/}{github}.

    \item{\textbf{Open-source}: \hhref{https://github.com/PolarNick239/ExternalSortingOnGPU}{Out-of-core merge sort} with GPU acceleration. \hhref{https://gist.github.com/PolarNick239/7819fb7722fab09b37ecaee77c82cf58}{96-bit 3D Morton code}. OpenCL \hhref{https://github.com/PolarNick239/OpenMeanShift}{implementation} of EDISON mean shift. \hhref{https://github.com/opencv/opencv/pull/6078}{Implemented} Python bindings for OpenCL algorithms in OpenCV. Contributions to OpenCV, PyOpenCL, jupyter qtconsole and others. GPU monitoring in i3pystatus.}

    \item{\textbf{Hackathons}}: six awards on hackatons. Two first places on \hhref{https://github.com/PolarNick239/HackathonDroneSwarm}{X-Mas Hack} (mission planner for drone swarm). Third place on \hhref{https://career.luxoft.com/lp/hack-cv/}{HackCV} (traffic signs recognition), \hhref{http://hackday.ru/sciencehackday-2/projects\#project-1400}{Science Hackday \#2} (Startup nomination), \hhref{http://hackday.ru/hackday-36/projects\#project-1121}{Hackday\#36} (Autodesk 3D-web nomination), \hhref{https://www.hackerleague.org/hackathons/jetbrains-edtech-hackathon/blogposts/53655896e24d32cfbd000006}{HackEdu} by JetBrains (third place). Participation in \hhref{http://www.hackjunction.com/}{Junction 2016, 2017}.

    \item{\textbf{Conferences}}: published \hhref{https://www.polarnick.com/static/papers/poliarnyi2021.pdf}{a paper} on \hhref{http://iccv2021.thecvf.com/}{ICCV 2021}. Participated in \hhref{http://3dv18.uniud.it/}{3DV 2018} and \hhref{http://www.3d-arch.org/}{3D-ARCH 2019}.

    \item{\textbf{Magister Ludi}}: \hhref{http://239.ru}{PML №239} programming teacher.
\end{itemize}

\vspace{-9pt}
\section*{\textbf{Education}}
\vspace{-9pt}

\begin{itemize}
    \item{Computer Science Center}
    \item{ITMO University, Computer Technologies}
    \item{PML №239, mathematical circle, programming contests}
\end{itemize}

\begin{wrapfigure}{r}{0.22\textwidth}
    \centering
    \includegraphics[width=0.22\textwidth]{unicorn.png}
\end{wrapfigure}

\vspace{-9pt}
\section*{\textbf{Contacts}}
\vspace{-9pt}

\begin{itemize}

    \item{\textbf{\hhref{mailto:PolarNick239@gmail.com}{PolarNick239@gmail.com}}}

    \item{\textbf{\hhref{http://polarnick239.github.io/index_ru.html}{PolarNick.ru}}}

    \item{\textbf{\hhref{https://github.com/PolarNick239}{GitHub}}}

    \item{\textbf{\hhref{https://www.linkedin.com/in/nikolai-poliarnyi-61393b7b}{LinkedIn}}}

\end{itemize}

Last updated: 24.03.2024

\end{document}
